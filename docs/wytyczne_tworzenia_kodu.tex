\documentclass[titlepage]{article}
\usepackage{hyperref}
\usepackage[polish]{babel}
\usepackage{polski}
\usepackage[utf8]{inputenc}
\usepackage{listings}
\usepackage{color}
\title{Wytyczne tworzenia kodu w projekcie GCS}
\author{Marcin Kolny\\
\texttt{marcin.kolny@gmail.com}}
\date{\today}
\begin{document}
\definecolor{cppred}{rgb}{0.6,0,0} % strings
\definecolor{cppgreen}{rgb}{0.25,0.5,0.35} % comments
\definecolor{cpppurple}{rgb}{0.5,0,0.35} % keywords
\definecolor{cppblue}{rgb}{0.25,0.35,0.75} % doxygen
\definecolor{lightgrey}{rgb}{0.9,0.9,0.9}
\lstset{language=C++,
basicstyle=\ttfamily,
keywordstyle=\color{cpppurple}\bfseries,
stringstyle=\color{cppred},
commentstyle=\color{cppgreen},
morecomment=[s][\color{cppblue}]{/**}{*/},
numbers=left,
backgroundcolor=\color{lightgrey},
numberstyle=\tiny\color{black},
stepnumber=1,
numbersep=10pt,
tabsize=4,
showspaces=false,
showstringspaces=false}
\pagestyle{plain}
\maketitle

\tableofcontents
\cleardoublepage
\section{Wybór nazw}
Wszystkie nazwy powinny być pisane w języku angielskim. Ponadto, należy stosować się do niżej wymienionych reguł.
\subsection{Zmienne}
Nazwy zmiennych pisane z małej litery. W przypadku nazw wieloczłonowych, każdy następny człon wyrazu rozdzielany znakiem podłogi \textbf{\_}.
\paragraph{}
\textbf{Przykład}
\begin{lstlisting}
int variable;
float long_name;
\end{lstlisting}
\subsection{Funkcje}
Nazwy funkcji pisane z małej litery. W przypadku nazw wieloczłonowych, każdy następny człon wyrazu rozdzielany jest znakiem podłogi \textbf{\_}.
\paragraph{}
\textbf{Przykład}
\begin{lstlisting}
bool compare(double x, double y );
void say_hello(string name);
\end{lstlisting}
\subsection{Klasy i struktury}
Nazwy klas i struktur pisane z wielkiej litery. W przypadku nazw wieloczłonowych, każdy następny człon pisany wielką literą (notacja \textit{PascalCase}).
\paragraph{}
\textbf{Przykład}
\begin{lstlisting}
class Collector
{
	...
};

struct SomeStruct
{
	...
};
\end{lstlisting}
\subsection{Przestrzenie nazw}
Nazwy przestrzeni nazw pisane z wielkiej litery. W przypadku nazw wieloczłonowych, każdy następny człon pisany wielką literą (notacja \textit{PascalCase}).
\paragraph{}
\textbf{Przykład}
\begin{lstlisting}
namespace CustomFunctions
{
	...
}
\end{lstlisting}
\paragraph{}
\textbf{Dodatkowe uwagi} \\
Cały kod tworzony na potrzeby koła naukowego powinien znajdować się w przestrzeni nazw \textit{HighFlyers}. Wskazane jest również zagnieżdżanie przestrzeni nazw tak, by opisywały one sekcję autorska. Nie jest to jednak konieczne.
\begin{lstlisting}
namespace HighFlyers
{
	namespace GCS
	{
		// GCS code...
	}

	// common code...
}
\end{lstlisting}
\subsection{Definicje}
Definicje pisane w całości wielkimi literami. W przypadku nazw wieloczłonowych, każdy następny wyraz oddzielony znakiem podłogi \textbf{\_}.
\paragraph{}
\textbf{Przykład}
\begin{lstlisting}
#define PI 3.14
#define EULERS_NUMBER 2.718
\end{lstlisting}
\subsection{Argumenty szablonowe}
Argumenty szablonowe piszemy wielkimi literami. W przypadku nazw wieloczłonowych, każdy następny wyraz oddzielony znakiem podłogi \textbf{\_}.
\paragraph{}
\textbf{Przykład}
\begin{lstlisting}
template<typename ARG, typename SECOND_ARG>
class SampleClass
{};

\end{lstlisting}

\section{Zapis kodu}
\subsection{Wcięcia}
Wcięcia w kodzie powinny być wstawiane w każdym zagnieżdżeniu fizycznego bloku w postaci dodatkowego znaku tabulacji.
\paragraph{}
\textbf{Przykład}
\begin{lstlisting}
class Example
{
	int v1;

	void Method( int arg )
	{
		vector<int> v1 = {1, 4, 7};

		v1 = 0;

		for (auto i : v1)
		v1 += i;
	}
};
\end{lstlisting}
\paragraph{}
\textbf{Sytuacja wyjątkowa}
Przy wielu zagnieżdżonych przestrzeniach, kod źródłowy może nie być czytelny. Dla tego w przypadku przestrzeni nazw, nie ma obowiązku tworzenia dodatkowych wcięć.
\begin{lstlisting}
namespace HighFlyers
{
namespace GCS
{
class TestClass
{
...
};
}
}
\end{lstlisting}

\subsection{Oddzielanie bloków kodu}
Blokiem w tym podpunkcie nazywany jest nie tylko zagnieżdżony w ramach jednego zagnieżdżenia, ale również logicznie pod względem przeznaczenia fragment kodu programu. Każdy blok powinien być oddzielony jedną pustą linią.
\paragraph{}
\textbf{Przykład}
\begin{lstlisting}
int x, y; // declaration's block

y = 3; // initialization's block

for (x = 2; x < 12; ++x) // loop's block
{
	y += x;
	cout << x << endl;
}

{
	// and some \"physical\" block
	int x = 12;
}
\end{lstlisting}
\paragraph{}
\textbf{Sytuacja wyjątkowa}
W przypadku bloków \textit{try/catch/finally} nie stosujemy przerwy w postaci pustej linii pomiedzy blokami.
\begin{lstlisting}
try
{
	if (y == 0)
		throw new Exception("Something is wrong...");
}
catch (Exception)
{
	y = 0.000001;
}
finally
{
	w = a / y;
}
\end{lstlisting}
\paragraph{}
\textbf{Dodatkowe uwagi}
Komentarze nie są traktowane jako puste linie w kodzie źródłowym. Poniższy fragment kodu ilustruje niepoprawne oraz poprawne przypadki użycia komentarzy.
\begin{lstlisting}
int y = -10;
// Error! there is no empty line between
// declaration's block and loop's block
while (y < 0)
	y++;


/*******************************/


int y = -10;

// OK! Empty line between two blocks
while (y < 0)
	y++;
\end{lstlisting}
\subsection{Nawiasy klamrowe}
Nawiasy klamrowe otwierane i zamykane w osobnych liniach. Ponadto, w linii z nawiasem klamrowym nie występuje już nic innego (ew. znaki tabulacji).

\paragraph{}
\textbf{Przykład}
\begin{lstlisting}
for (int x = 0; x < 10; x++)
{
	z += 5;
	y += 10;
}
\end{lstlisting}
Kiedy nawiasy klamrowe nie są potrzebne, nie trzeba(ale można) ich wstawiać.

\subsection{Dodatkowe spacje w kodzie}
Każdy operator powinien być oddzielony od literału jedna spacja. Ponadto w przypadku pętli/warunków/instrukcji switch pomiędzy słowem kluczowym a nawiasem otwierającym umieszczana jest jedna spacja. Pomiędzy znakiem początku/końca komentarza a treścią komentarza występuje jedna spacja.
\paragraph{}
\textbf{Przykład}
\begin{lstlisting}
int h = w + 5;

if (x == 12)
{
	return true;
}

while (z < 10)
{
	// comment
}

switch (a)
{
	case 1:
		/* second comment */
		break;
}
\end{lstlisting}
W przypadku operatora inkrementacji / dekrementacji, operator nie jest oddzielony spacja od literału.
\section{Komentarze}
Komentarze są przydatnym elementem kodu, mającym na celu wyjaśnienie danego fragmentu programu. Nie należy jednak ich nadużywać. Poniżej wymienione zostały przypadki, w których należy stosować komentarze. 
Wszystkie komentarze pisane są w formacie Doxygen: \href{http://doxygen.org/}{dokumentacja Doxygen}.
Obowiązkowo komentarze powinny wystąpić dla następujących elementów:
\begin{itemize}
\item funkcje,
\item klasy,
\item struktury,
\item pola klas i struktur,
\item wartości globalne,
\item mało oczywiste, zawile fragmenty programu (niekoniecznie w formacie doxygen).
\end{itemize}

\subsection{Język komentarzy}
Komentarze pisane w języku angielskim.
\section{Uwagi i propozycje zmian}
Wszelkie uwagi oraz propozycje zmian należy zgłaszać autorowi dokumentu na adres e-mail \href{mailto:marcin.kolny@gmail.com}{marcin.kolny@gmail.com}
\end{document}
