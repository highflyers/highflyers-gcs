\documentclass[a4paper,12pt]{article}
\usepackage{polski}
\usepackage[utf8]{inputenc}
\usepackage{hyperref}
\usepackage[margin=30mm]{geometry}
\usepackage{fancyhdr}
\pagestyle{fancy}
\usepackage{graphicx}

%%%%%%%%%%%%%% TUTAJ USTAWIENIA %%%%%%%%%%%%%%%%%%%
\author{Marcin Kolny}
\newcommand{\reportnumber}{1} % numer raportu/spotkania
\newcommand{\teamname}{GCS} % nazwa grupy
\date{14 listopada 2013} % data odbycia spotkania
%%%%%%%%%%%%%% KONIEC USTAWIEN %%%%%%%%%%%%%%%%%%%

\newcommand{\reportnumberfull}{\reportnumber/\the\year}
\begin{document}
\rhead{zebranie nr \reportnumberfull}
\lhead{
  \begin{picture}(0,0)
    \put(0, 20){\raisebox{-0.5\height}{\includegraphics[width=20mm]{logo.png}}}
  \end{picture}
}
\begin{minipage}{0.46\textwidth}
  \begin{flushleft}
    \textbf{MKN HIGH FLYERS}\\
    \teamname
  \end{flushleft}
\end{minipage}
\begin{minipage}{0.46\textwidth}
  \begin{flushright}
    \makeatletter
    \@date
    \makeatother
  \end{flushright}
\end{minipage}\\
\begin{center}
  \textsc{\LARGE SPRAWOZDANIE Z ZEBRANIA\\NR \reportnumberfull}
\end{center}


\begin{tabular}{ |p{0.5cm}|p{6cm}| }
  \hline
  \multicolumn{2}{|c|}{Lista obecnych} \\
  \hline
  1 & Marcin Kolny\\
  2 & Wojciech Dudzik\\
  3 & Tomasz Widenka\\
  4 & Maciej Smolarczyk\\
  5 & Tomasz Guźniczak\\
  6 & Mateusz Ucher\\
  \hline
\end{tabular}

\section{Sprawy ogólne}
\begin{enumerate}
\item{Licencja GCS}\\
  Póki co nie ustalono licencji, na jakiej będzie wydawane oprogramowanie. Ustalono natomiast następujące warunki licencji:
  \begin{enumerate}
  \item{Niewirusowa}
  \item{Możliwość modyfikacji przez każdą osobę}
  \item{Możliwość używania w komercyjnych projektach}
  \end{enumerate}
  Na następnym spotkaniu zostanie ustalona licencja.
\item{Aktualizacja pliku README.md}\\
  Aktualizacji pliku podjął się Mateusz Ucher, z pomocą Marcina Kolnego. Plik powinien zawierać kilka słów o projekcie, a także instrukcję kompilacji projektu.
\item{Nowy podprojekt - symulator}\\
  Sprawa na razie oczekuje na realizację, zajmie się prawdopodobnie Tomasz Widenka.
\item{Ustalenia odnośnie zapisu kodu}\\
  Zmiany odnoszące się do stylu zapisu kodu zostały zapisane w commicie: highflyers-gcs : master : 1160eeacd6158f077c65883d06938be25ea1a8f0
\item{Nagłówki plików}\\
  Nie zostało ustalone, co powinno być w nagłówkach każdego pliku. Będzie to zależne od wybranej licencji.
\item{Hardware dla GCS}\\
  Paulina Wilk rozgląda się za sprzętem, mogącym posłużyć jako Naziemna Stacja Kontroli Lotu.
\end{enumerate}

\section{Sprawy techniczne}
\begin{enumerate}
\item{Wyjątki}\\
  Zdecydowaliśmy, że ze względu na rozbudowaną hierarchię wyjątków w bibliotece standardowej, nie będziemy definiować własnych klas.
\item{H vs HPP}\\
  Zostały ustalone zasady związane z rozszerzeniami dla plików. highflyers-gcs : master : 1160eeacd6158f077c65883d06938be25ea1a8f0
\end{enumerate}
\begin{flushright}
  \line(1,0){120}\\
  Autor sprawozdania:\\
  \makeatletter
  \emph{\@author}
  \makeatother
\end{flushright}

\end{document}

